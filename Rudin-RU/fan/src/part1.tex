%! Author = artemsemidetnov
%! Date = 17.06.2022

% Preamble
\documentclass[11pt]{article}

% Packages
\usepackage{amsmath}

% Document
\begin{document}

Приведем важное замечание, которым потом будем много раз пользоваться
\begin{remark}
Для всякой окрестности нуля $W$ существует такая симметричная относительно нуля окрестность $U$, что $U+U \subseteq W$.
\end{remark}
\begin{proof}
Сложение непрерывно $\Rightarrow$ найдутся $V_1, V_2$ -- окрестности нуля, что $V_1 + V_2 \subseteq W$. Положим
    $$U = V_1 \cap V_2 \cap (-V_1) \cap (-V_2)$$
\end{proof}
\begin{theorem}
 $K$ -- компакт, $C$ -- замкнутое и $K\cap C = \varnothing$ $\Rightarrow$ существует окрестность нуля $V$:
    $$(K+V) \cap (C+V) = \varnothing$$
\end{theorem}
\begin{proof}
Пусть $K \neq \varnothing$, каждой его точке $x$ по предыдущему замечанию сопоставим симметричную окрестность нуля $V_x$ такую, что $x+ V_x+V_x +V_X\cap C = \varnothing$. Из симметричности
    $$\left( x+V_x+V_x\right) \cap \left( C+V_x\right)$$
    Поскольку $K$, компакт, хватит конечного количества таких окрестностей, чтобы его покрыть, то есть $\exists x_1,\ldots, x_n$:
    $$K \subseteq \left(x_1+V_{x_1}\right) \cup \ldots \cup \left(x_n+V_{x_n}\right)$$
    Тогда возьмем $V = V_{x_1} \cap \ldots \cap V_{x_n}$
\end{proof}

Из этого замечания можно извлечь несколько полезных следствий: \\
Если взять $K = \{0\}$, получим
\begin{theorem}
    Пусть $\mathcal{B}$ -- локальная база $X$. Тогда для всякой $V_1 \in \mathcal{B}$ существует $V_2 \in \mathcal{B}$ такая, что $\overline{V_2} \subseteq V_1$
\end{theorem}
Если взять $K = \{x\}, C = \{y\}$, получим
\begin{theorem}
    Топологические векторные пространства Хаусдорфовы
\end{theorem}


\begin{theorem}\label{urav}
    В окрестности нуля всегда лежит уравновешенная окрестность нуля. В выпуклой окрестности нуля всегда лежит выпуклая уравновешенная окрестность нуля.
\end{theorem}
\begin{proof}Пусть $V$ -- окрестность нуля. \\
    По непрерывности умножения, найдутся такие $\delta > 0$ и окрестность нуля $U$, что $\alpha U \subseteq V$ при $|\alpha| < \delta$.
    Объединим все такие $\alpha U$ и получим уравновешенную окрестность нуля
\end{proof}

Заметим, что предыдущая теорема означает следующее:
\begin{remark}
Каждое ТВП обладает уравновешенной локальной базой. Каждое локально выпуклое ТВП обладает уравновешенной выпуклой локальной базой.
\end{remark}
Поговорим теперь про раздутие окрестностей:
\begin{remark}
Пусть $V$ -- окрестность нуля в ТВП $X$, $r_n \to \infty, \delta_n \to 0$, причем $r_n$ -- положительные и монотонно возрастают, а $\delta_n$ -- положительные и монотонно убывают. Тогда
    $$X = \bigcup_{n = 1}^\infty r_nV$$
    и $$\{\delta_n V\}_{n = 1}^\infty \quad \text{является локальной базой }X$$
\end{remark}
\begin{proof}

\end{proof}
Первое замечание, которое надо сделать заключается в том, что непрерывность \textbf{линейного оператора} достаточно и необходимо проверять в нуле.
Дальше надо отметить следующий факт:
\begin{remark}
Свойства выпуклости, линейности, уравновешенности сохраняются при взятии образов и прообразов линейных отображений.
\end{remark}
Что-то определенное можно сказать в общем случае про функционалы (операторы действующие в поле скаляров)
\begin{theorem}
    Пусть $\Lambda$ -- нетривиальный функционал на ТВП $X$. Следующие условия эквивалентны:
    \begin{enumerate}
        \item $\Lambda$ непрерывен
        \item $\ker \Lambda$ замкнуто
        \item $\ker \Lambda$ не плотно
        \item $\Lambda$ финитно ограничен в нуле
    \end{enumerate}
\end{theorem}
\begin{proof}\ \\
\begin{enumerate}
    \item[$(1\Rightarrow 2)$] Очевидно
    \item[$(2\Rightarrow 3)$] Очевидно
    \item[$(3 \Rightarrow 4)$] Если $\ker \Lambda$ не плотно, существуют $x \in X$ и  уравновешенная (по \ref{urav}) окрестность нуля $V$ такая, что
        $(x+V) \cap \ker \Lambda = \varnothing$. При этом $\Lambda V$ -- уравновешенное множество в $F$. Что в $\C$, что в $\R$ мы такие знаем. Они либо ограничены (а тогда выполнено нужное нам условие).
    Либо это всё $F$, однако в этом случае, очевидно, $x + V \cap \ker \Lambda \neq \varnothing$.
    \item[$(4 \Rightarrow 1)$] Пусть $V$ -- такая окрестность нуля, что $|\Lambda x|<M$ на $V$, отсюда $|\Lambda x|<r$ на $\frac{r}{M}V$, а значит функционал непрерывен.
\end{enumerate}

\end{proof}

\end{document}