\documentclass[12pt, a4paper, oneside]{book}
\usepackage[no-chapters, fix-math-spacing, slanted-inequalities]{preamble}
\usepackage{files/lua}
\usepackage{amssymb}
\usepackage{stmaryrd}

\begin{document}

\SupplKernOn
%\input{title}

\frontmatter

\tableofcontents
\pagebreak

%%%%%%%%%%%%%%%%%%%%%%%%%%%%%

\mainmatter

\section{Вступление}
Во всем конспекте $F$ обозначает поле скаляров, все векторные пространства будем смотреть только над ним.
В качестве $F$ мы берем только $\R$, либо $\C$ с естественными топологиями на них.

\begin{definition}
    Топологическое пространство $X$ являющееся векторным пространством называется
    топологическим векторным пространством (ТВП), если
    \begin{enumerate}
        \item Топология удовлетворяет $T_1$ (синглетоны замкнуты)
        \item Сложение и умножение на скаляр непрерывны
    \end{enumerate}
\end{definition}

\begin{remark}\ \\
    \begin{enumerate}
        \item Сдвиг на любой вектор $ u \mapsto u+v$ и растяжение на любой скаляр $\alpha \neq 0: \,u \mapsto \alpha u $ являются гомеоморфизмами.
        \item Топология для такого $X$ всегда инвариантна относительно сдвигов.
        \item Следовательно, полностью определяется локальной базой.

    \end{enumerate}
\end{remark}
\ \\
Это основное определение, помимо него напомним еще определений:
\begin{definition}
    Пусть X -- векторное пространство и $A \subseteq X$
    \begin{enumerate}
        \item Если $0\in A$ и  $\forall \alpha,\beta \in F$ выполняется $\alpha A + \beta A \subseteq A$, то $A$ называется \textbf{подпространством} X; обозначается $A \leqslant X$.
        \item Если $\forall t \in (0,1)$ выполняется $tA + (1-t)A \subseteq A$, то $A$ называется \textbf{выпуклым}.
        \item Если $\forall \alpha \in F: \; |\alpha| \leqslant 1$ выполняется $\alpha A \subseteq A$, то $A$ называется \textbf{уравновешенным}.
    \end{enumerate}
\end{definition}

\begin{remark}
Подпространства являются выпуклыми уравновешенными множествами.
\end{remark}

\begin{definition}[Типы пространств]
    Пусть $X$ -- ТВП, говорим, что $X$
    \begin{enumerate}
        \item[(A)] \textbf{локально выпукло}, если существует локальная база из выпуклых окрестностей.
        \item[(B)] \textbf{локально ограничено}, если существует ограниченная окрестность нуля.
        \item[(C)] \textbf{локально компактно}, если существует предкомпактная окрестность нуля.
        \item[(D)] \textbf{метризуемо}, если его топология метризуема.
        \item[(F)] является \textbf{F-пространством}, если топология индуцируется \textbf{полной инвариантной} метрикой.
        \item[(G)] является \textbf{пространством Фреше}, если оно локально-выпуклое $F$-пространство
        \item[(E)] \textbf{нормируемо}, если существует норма, индуцирующая топологию $X$
        \item[(F)] обладает свойством \textbf{Гейне-Бореля}, если
        \begin{center}(ограниченное $\wedge$ замкнутое $\Rightarrow\,$ компактное)
            \end{center}
    \end{enumerate}
\end{definition}

Небольшой обзор результатов касающихся затронутых выше понятий
\begin{theorem}[Воспоминания о будущем]\ \\
\begin{enumerate}
    \item Локально ограничено $\Rightarrow$ обладает счетной локальной базой.
    \item Метризуемо $\Leftrightarrow$ обладает счетной локальной базой.
    \item Нормируемо $\Leftrightarrow$ (локально выпукло $\wedge$ локально ограничено)
    \item Конечномерно  $\Leftrightarrow$ локально компактно
    \item Обладает свйоством Гейне-Бореля $\Rightarrow$ конечномерно
\end{enumerate}
\end{theorem}

\subsection{Про ограниченные множества}
Существует два различных определения ограниченных множеств, в которых легко запутаться, если речь идет о метризуемых ТВП:

\begin{minipage}[t]{50mm}\parindent=2em
\begin{center}
$(X,d)$ -- \textbf{метрическое пространство}
\end{center}
    $E \subset X$ называется ограниченным, если $\exists M > 0$ :\\
    $$\forall x, y \in E \; \; d(x,y) < M$$
\end{minipage}
\hfill
\begin{minipage}[t]{50mm}
\begin{center}
   $X$ -- \textbf{ТВП}
\end{center}
    $E \subset X$ называется ограниченным, если \\
    $$\forall U \text{ -- окр.} 0 \; \exists t_0 > 0 : \; \forall t > t_0 $$
$$ E \subseteq tU $$
\end{minipage}
\begin{lemma}
Если $X$ нормируемое пространство и $d(x, y) = ||x-y||$, то эти понятия совпадают. В ином случае они могут отличаться во всех нетривиальных случаях.
\end{lemma}
\begin{proof}
Почему они отличаются в других случаях: достаточно взять новую инвариантную метрику, индуцирующую ту же самую топологию $d' = d/(1+d)$. В ней всё $X$ будет ограниченным.

    Пусть $(X, || \cdot ||)$ -- нормированное пространство и $d(x,y) = ||x-y||$. За $B_R$ будем обозначать шар радиуса $R$ в нуле.
    Понятно, что достаточно показать, что $B_1$ -- топологически ограничен (Если $E$ метрически ограниченно $\Rightarrow$ $E \subseteq B_R = R B_1$).
    Так как топология $X$ индуцируется метрикой, базой в точке выступают шары, а значит какой-то $B_\varepsilon = \varepsilon B_1$ содержится в $U$.

    Пусть теперь $E$ топологически ограничено, предположим, что оно не метрически ограничено.
    То есть, существует $\{x_n\} \subset E$ такая, что $||x_n|| \to +\infty$. Тогда в качестве окрестности нуля попробуем взять $B_1$, понятно, что $\sup_{x \in tB_1} ||x|| = t$, а значит, не найдется $t_0$ подходящего под условия.
\end{proof}



\subsection{Пространства $C^\infty(\Omega), \cD_K$}
Пусть $\Omega \subseteq \R^n$ -- открытое множество. С мультиииндексом $\alpha = (\alpha_1, \ldots, \alpha_n)$ свяжем дифференциальный оператор
$$D^\alpha = \left(\frac{\partial}{\partial x_1}\right)^{\alpha_1}\ldots \left(\frac{\partial}{\partial x_n}\right)^{\alpha_n}.$$
Порядком $\alpha$ назовем число $|\alpha| = \alpha_1+\ldots+\alpha_n$ и доопределим $D^\alpha f = f$, если $|\a| = 0.$ Пусть $K\subseteq \Omega$ -- компакт. Определим множество
$$\cD_K = \{ f \in C^\infty(\Omega) : \supp f \subseteq K\}$$
Представим $\Omega = \cup K_i$, где $K_i$ -- компакты и $K_i \subset \Int K_{i+1}$








\section{О полноте}
Под $\Hom(X,Y)$ обозначается множество всех линенйных отображений $X \to Y$, под $\CHOM(X,Y)$ множество тех из них, которые непрерывны в тополгиях $X,Y$.

\begin{theorem}[Бэра о полноте]
    В полных метрических пространствах и локально компактных хаусдорфовых пространствах
    $$\bigcap_{n \in \N} (\text{открытое, всюду плотное}) \quad \textbf{ -- всюду плотно}$$
\end{theorem}

\subsection{Теорема Банаха-Штейнгауза}
Сначала, введем определение
\begin{definition}
    Пусть $\G \subseteq \Hom(X,Y)$. Оно называется равностепенно непрерывным, если
    $$\text{для любой }\; W - \text{окр. 0 в Y существует }\; V - \text{окр. 0 в X: } \; \Gamma(V) \subseteq W$$
\end{definition}

Как мы увидим ниже, равностепенно-непрерывные семейства переводят ограниченные множества в ограниченные.
Теорема Банаха-Штейнгауза же скажет, что если точек, орбиты которых под действием $\Gamma$ ограниченны {\it много}, то $\Gamma$ равностепенно непрерывно.

\begin{theorem}
    Пусть $\Gamma \subseteq \Hom(X,Y)$ -- равностепенно непрерывно, а $E \subseteq X$ ограниченно.
    Тогда $\Gamma(E)$ суть ограниченное множество в $Y$.
\end{theorem}
\begin{proof}
Рассмотрим $W$ -- окрестность 0 в $Y$, выберем $V$ -- окрестность 0 в $X$ из определения. $E$ -- ограниченно $\Rightarrow$ имеем
$E \subseteq tV$ для больших $t$. Для них же:
    $$\Gamma(E) \subseteq \Gamma(tV) = t\Gamma(V) \subseteq t W$$
\end{proof}

\begin{theorem}[Банаха-Штейнгауза]
    Пусть $\Gamma \subseteq \CHOM(X,Y)$. Предположим, что множество $B = \{x \in X \; : \; \Gamma(x) - \text{ограниченно}\}$ имеет вторую категорию в $X$.
\begin{center}
    Тогда $B = X$ и $\Gamma$ равностепенно непрерывно.
\end{center}

\end{theorem}
\begin{proof}
Пусть $W$ -- уравновешенная окрестность 0 в $Y$, будем искать такую $V$ в $X$, что $\Gamma(V) \subseteq W$.
Для этого найдем такую $U$ -- уравновешенная окрестность 0 в $Y$, что $\overline{U} + \overline{U} \subseteq W,$ положим
$$E = \cap\limits_{\Lambda \in \Gamma}\Lambda^{-1}\big(\overline{U}\big).$$

$$x \in B \Rightarrow \Gamma(x) \in nU \text{ для больших }n \Rightarrow \frac{1}{n}x \in E \text{ для таких же }n$$
    Тогда заметим, что $B \subseteq \cup_n nE$. Значит, какой-то из $nE$ --
множество второй категории. Заметив, что $x \mapsto nx $ это гомеоморфизм $X$ получаем:
    $$\begin{rcases*}
E - \text{множество второй категории} \\
E - \text{замкнуто так как все $\Lambda \in \Gamma$ непрерывны}
\end{rcases*} \Rightarrow \text{в $E$ есть внутренняя точка } x_0$$
    Значит, в множестве $x_0 - E$ содержится окрестность нуля $V$, причем:
    $$\Lambda(V) \subset \Lambda x_0 - \Lambda(E) \subseteq \overline{U} - \overline{U} \subseteq W$$
    Тогда $\Gamma$ переводит ограниченные множества в ограниченные, а тогда $\Gamma(x)$ -- ограниченное для всякого $x \; \Rightarrow B = X$.
\end{proof}
\subsection{Полезные частные случаи}
\begin{theorem}
    Пусть $X$ -- $F$-пространство и $\Gamma \subseteq \CHOM(X,Y)$ и $\forall x \in X \; \Gamma(x)$ ограниченно в $Y$.
    Тогда $\Gamma$ равностепенно непрерывно.
\end{theorem}
\begin{proof}
По теореме Бэра, $F$-пространства являются множествами второй категории в себе.
\end{proof}
\begin{theorem}
    Пусть $X$ -- банахово, $Y$ -- нормируемо, причем $\sup_{\Lambda \in \Gamma} ||\Lambda x || < \infty$. Тогда существует такой $M > 0$, что
    $$||\Lambda x|| \leqslant M ||x||\quad \forall x \in X \; \forall \Lambda \in \Gamma$$
\end{theorem}
\begin{proof}
Применим предыдущую теорему к метрикам, порожденным нормами. В них ограниченность эквивалентна метрической.
\end{proof}
\begin{theorem}
    Пусть $\Lambda_n \in \CHOM(X,Y)$. Определим $$ C = \{x \in X\; : \; \Lambda_n x \text{ -- последовательность Коши}   \}; \quad  L = \{x \in X \; : \; \Lambda_n x \to \Lambda x \}$$.
    Тогда
    \begin{enumerate}
        \item Если $C$ второй категории в $X$, то $C = X$
        \item Если $L$ второй категории в $X$ и $Y$ -- $F$-пространство, то $L = X$ и $\Lambda$ -- непрерывно.
    \end{enumerate}
\end{theorem}
\begin{proof}\ \\
\begin{enumerate}
    \item Все последовательности Коши ограниченны $\Rightarrow $ по Б-Ш $\{\Lambda_n\}$ равностепенно непрерывна.
    Заметим, что $C \leqslant X$. Также, $C$ -- всюду плотное. \big({\small Пусть не так $\Rightarrow$ тогда $X = \overline{C} \oplus Y$, то есть $\overline{C}$ -- собственное подпространство.
    $\Rightarrow$ в нем нет внутренних точек $\Rightarrow$ оно I категории}\big).
    \ \\ Возьмем $x \in X$, будем показывать, что $\Lambda_n x$ -- последовательность Коши в $Y$, зафиксируем $W$ -- окрестность 0 в $Y$.
    По равностепенной непрерывности найдем симметричную $V$ -- окрестность 0 в $X$ -- такую, что ВСЕ $\Lambda_n(V) \subseteq W$.
    Так как $C$ всюду плотно, найдем $y \in C \cap \left(x+V \right)$. Тогда НСНМ
    $$(\Lambda_n - \Lambda_m)x = \Lambda_n(x-y) + (\Lambda_n - \Lambda_m)y + \Lambda_m(y-x) \in W + W + W$$
    \item
\end{enumerate}
\end{proof}

\subsection{Теорема об открытом отображении}
Пусть $f:S \to T$ -- отображение {\footnotesize ($S,T$ -- топологические, $f$ не обязательно непрерывна)} и $p \in S$.
Говорят, что $f$ открыта в $p$, если для любой $V$ -- окрестности $p$ образ $f(V)$ содержит окрестность $f(p)$.
\begin{theorem}
   Пусть $X$ -- $F$-пространство, $\Lambda \in \CHOM(X,Y)$, причем $\Lambda(X)$ второй категории. Тогда\\
     $\Lambda(X) = Y$, \; \; $\Lambda$ открыто,\; \; $Y$ - $F$-пространство
\end{theorem}
\begin{proof}
Заметим, что из второго следует первое, поскольку $\im \Lambda$ это подпространство. Докажем второе.

Пусть $V$ -- окрестность 0 в $X$, мы хотим проверить, что $\Lambda(V)$ содержит окрестность 0 в $Y$. Заведем полную инвариантную метрику $d$ на $X$.
    Определим $$V_n = \left\{x\; | \; d(x,0) < \frac{r}{2^n}\right\}$$
    где $r$ такой маленький, что $V_0 \subset V$. Мы будем показывать, что $\overline{\Lambda(V_1)}\subseteq V$ и $W \subseteq \Lambda(V_1)$ для некоторой окрестности нуля $W$ в $Y$.
    Знаем, что $$\overline{\Lambda(V_1)} \supseteq \overline{\Lambda(V_2) - \Lambda(V_2)} \supseteq \overline{\Lambda(V_2)} - \overline{\Lambda(V_2)}$$
    так как $V_2 - V_2 \subseteq V_1$.
\end{proof}


\subsection{Теорема о замкнутом графике}
Под графиком отображения $f: X \to Y$ имеется ввиду множество $\{(x,f(x))\}_{x \in X} \subseteq X \times Y$. Для непрерывных отображений в хаусдорфовы пространства график всегда замкнут, мы будем пытаться выяснить про какие-то факты, похожие на это.
Для начала обоснуем факт про замкнутность графика непрерывной функции:
\begin{remark}
Пусть $f:X \to Y$ непрерывна и $Y$ -- Хаусдорфово. Тогда график $f$ замкнут.
\end{remark}
\begin{proof}
Рассмотрим $(x_0, y_0)$ из дополнения графика в $X \times Y$, тогда отделим по хаусдофовости в $Y$ точки $y_0, f(x_0)$ окрестностями $U, V$ соответственно.
По непрерывности найдем $W$ -- окрестность $x_0$ в $X$ такую, что $f(W) \subseteq V$. А значит, $W \times U$ -- искомая окрестность $(x_0, y_0)$, содержащаяся в дополнении графика.
\end{proof}

\begin{theorem}[О замкнутом графике]
    Пусть $X,Y$ -- $F$-пространства, $\Lambda \in \Hom(X,Y)$ и его график замкнут. Тогда $\Lambda$ непрерывен.
\end{theorem}

\begin{proof}
    Рассмотрим $X\times Y$ как $F$-пространство.
График $\Lambda$ -- обозначим его $G$ -- замкнутое подпространство в $X\times Y$ (поскольку лямбда линейна). А значит, $G$ само по себе $F$-пространство
    Определим $$\pi_1: G \to X\; (x,\Lambda x) \mapsto x$$
    $$\pi_2: X\times Y \to Y\; (x,y) \mapsto y$$
    Тогда $\pi_1$ непрерывная линейная биекция $G\to X$, причем $G$ и $X$ -- $F$-пространства. Тогда по теореме об открытом отображении $\pi_1^{-1}$ непрерывно. А значит,
     $$\Lambda = \pi_2 \circ \pi_1^{-1} \; \text{непрерывна}$$
\end{proof}

\begin{remark}
Пусть для всяких $x_n \to x, \Lambda x_n \to y$ выполняется $y = \Lambda x$. Тогда график $\Lambda$ замкнут.
\end{remark}

\subsection{Билинейные отображения}
Пусть $X,Y,Z$ -- ТВП и $B: X\times Y \to Z$.

Можем определить $B_x:Y\to Z,\, B^y:X \to Z$ для фиксированных $x,y$ -- функции на срезах $X\times Y$.  Если они непрерывны, то $B$ называется \textbf{раздельно непрерывным}; если $B_x, \, B^y$ линейны, то $B$ называется \textbf{билинейным}.
В некоторых случаях из раздельной непрерывности следует обычная непрерывность:
\begin{theorem}
    Если $X$ -- $F$-пространство и $B$ -- раздельно непрерывное билинейное. Тогда $B$ секвенциально непрерывно. В частности, если $Y$ метризуемо, то $B$ непрерывно.
\end{theorem}
\begin{proof}
    Выберем $x_n \to x_0$ в $X$ и $y_n \to y_0$ в $Y$
Возьмем $U, W$ окрестности 0 в $Z$ такие, что $U+U \subseteq W$, положим $b_n(x) = B(x,y_n)$.
    \begin{enumerate}
        \item  Так как эти последовательности сходятся, множества $\{b_n(x)\}$ ограничены в $Z$.
        \item  Тогда $b_n(x)$ --  непрерывные линейные отображения из $F$-пространства $X$ в $Z$.
        \item  Значит, по \ref{}[следствию из теоремы Банаха-Штейнгауза], семейство $\{b_n\}$ равномерно непрерывно.
    \end{enumerate}
    А значит найдется $V$ -- окрестность 0 в $X$ такая, что $\forall n\; b_n(V) \subset U$. Тогда начиная с некоторого места:
    $$B(x_n, y_n) - B(x_0,y_0) = b_n(x_n-x_0)+B(x_0,y_n-y_0) \in U + U \subseteq W$$
    Если $Y$ метризуемо, то $X\times Y$ метризуемо, а значит секвенциальная непрерывность эквивалентна обычной
\end{proof}
    \bibliography{main}
    \bibliographystyle{plain}

\end{document}
