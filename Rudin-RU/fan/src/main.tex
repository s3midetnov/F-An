\documentclass[12pt, a4paper, oneside]{book}
\usepackage[no-chapters, fix-math-spacing, slanted-inequalities]{preamble}
\usepackage{files/lua}
\usepackage{amssymb}
\usepackage{stmaryrd}
\usepackage{wasysym}

\begin{document}

\SupplKernOn
%\input{title}

\frontmatter

\tableofcontents
\pagebreak

%%%%%%%%%%%%%%%%%%%%%%%%%%%%%

\mainmatter

\section{Вступление}
Во всем конспекте $F$ обозначает поле скаляров, все векторные пространства будем смотреть только над ним.
В качестве $F$ мы берем только $\R$, либо $\C$ с естественными топологиями на них.

\begin{definition}
    Топологическое пространство $X$ являющееся векторным пространством называется
    топологическим векторным пространством (ТВП), если
    \begin{enumerate}
        \item Топология удовлетворяет $T_1$ (синглетоны замкнуты)
        \item Сложение и умножение на скаляр непрерывны
    \end{enumerate}
\end{definition}

\begin{remark}\ \\
    \begin{enumerate}
        \item Сдвиг на любой вектор $ u \mapsto u+v$ и растяжение на любой скаляр $\alpha \neq 0: \,u \mapsto \alpha u $ являются гомеоморфизмами.
        \item Топология для такого $X$ всегда инвариантна относительно сдвигов.
        \item Следовательно, полностью определяется локальной базой.

    \end{enumerate}
\end{remark}
\ \\
Это основное определение, помимо него напомним еще определений:
\begin{definition}
    Пусть X -- векторное пространство и $A \subseteq X$
    \begin{enumerate}
        \item Если $0\in A$ и  $\forall \alpha,\beta \in F$ выполняется $\alpha A + \beta A \subseteq A$, то $A$ называется \textbf{подпространством} X; обозначается $A \leqslant X$.
        \item Если $\forall t \in (0,1)$ выполняется $tA + (1-t)A \subseteq A$, то $A$ называется \textbf{выпуклым}.
        \item Если $\forall \alpha \in F: \; |\alpha| \leqslant 1$ выполняется $\alpha A \subseteq A$, то $A$ называется \textbf{уравновешенным}.
    \end{enumerate}
\end{definition}

\begin{remark}
Подпространства являются выпуклыми уравновешенными множествами.
\end{remark}

\begin{definition}[Типы пространств]
    Пусть $X$ -- ТВП, говорим, что $X$
    \begin{enumerate}
        \item[(A)] \textbf{локально выпукло}, если существует локальная база из выпуклых окрестностей.
        \item[(B)] \textbf{локально ограничено}, если существует ограниченная окрестность нуля.
        \item[(C)] \textbf{локально компактно}, если существует предкомпактная окрестность нуля.
        \item[(D)] \textbf{метризуемо}, если его топология метризуема.
        \item[(F)] является \textbf{F-пространством}, если топология индуцируется \textbf{полной инвариантной} метрикой.
        \item[(G)] является \textbf{пространством Фреше}, если оно локально-выпуклое $F$-пространство
        \item[(E)] \textbf{нормируемо}, если существует норма, индуцирующая топологию $X$
        \item[(F)] обладает свойством \textbf{Гейне-Бореля}, если
        \begin{center}(ограниченное $\wedge$ замкнутое $\Rightarrow\,$ компактное)
            \end{center}
    \end{enumerate}
\end{definition}

Небольшой обзор результатов касающихся затронутых выше понятий
\begin{theorem}[Воспоминания о будущем]\ \\
\begin{enumerate}
    \item Локально ограничено $\Rightarrow$ обладает счетной локальной базой.
    \item Метризуемо $\Leftrightarrow$ обладает счетной локальной базой.
    \item Нормируемо $\Leftrightarrow$ (локально выпукло $\wedge$ локально ограничено)
    \item Конечномерно  $\Leftrightarrow$ локально компактно
    \item Обладает свйоством Гейне-Бореля $\Rightarrow$ конечномерно
\end{enumerate}
\end{theorem}

\subsection{Про ограниченные множества}
Существует два различных определения ограниченных множеств, в которых легко запутаться, если речь идет о метризуемых ТВП:

\begin{minipage}[t]{50mm}\parindent=2em
\begin{center}
$(X,d)$ -- \textbf{метрическое пространство}
\end{center}
    $E \subset X$ называется ограниченным, если $\exists M > 0$ :\\
    $$\forall x, y \in E \; \; d(x,y) < M$$
\end{minipage}
\hfill
\begin{minipage}[t]{50mm}
\begin{center}
   $X$ -- \textbf{ТВП}
\end{center}
    $E \subset X$ называется ограниченным, если \\
    $$\forall U \text{ -- окр.} 0 \; \exists t_0 > 0 : \; \forall t > t_0 $$
$$ E \subseteq tU $$
\end{minipage}
\begin{lemma}
Если $X$ нормируемое пространство и $d(x, y) = ||x-y||$, то эти понятия совпадают. В ином случае они могут отличаться во всех нетривиальных случаях.
\end{lemma}
Доказательство очевидное, приведем пример, почему эти понятия различаются в иных случаях. Если $d$ -- любая метрика на $X$, то $d' = \frac{d}{1+d}$ -- тоже метрика,
причем эквивалентная изначальной. Однако относительно нее всё пространство (и все подмножества) будут ограничены.

\subsection{В Рудине такого не было!}
Категория ТВП -- $\TVect$.
\begin{enumerate}
    \item $\Ob(\TVect)$ -- топологические векторные пространства
    \item $X \to Y$ это непрерывные линейные отображения.
\end{enumerate}
Дисклеймер: не все стрелки в этом конспекте это морфизмы в категории.

Заметим, что $\TVect$ -- конкретная категория, поэтому определен обычный забывающий функтор $F_S:\TVect \to \Set$.  Tакже определим $F_V : \TVect \to \Vect$ -- забывающий топологию функтор.

Если $X,Y \in \Ob(\TVect)$, то $\Hom(X,Y)$ обозначает множество стрелок в $\Vect$ между $F_V(X)$ и $F_V(Y)$. Множетво стрелок в $\TVect$ мы обозначаем как $\CHOM(X,Y)$.

Во всем конспекте буквы $X,Y, Z$ обозначают элементы $\Ob(\TVect)$.



\section{О полноте}


\begin{definition}
    Говорят, что $A \subset X$ имеет $I$ категорию, если $A = \cup_n E_n$, где каждое из $E_n$ нигде не плотно.
    $A$ имеет вторую $II$ категорию в ином случае.
\end{definition}

\begin{theorem}[Бэра о полноте]
    В полных метрических пространствах и локально компактных хаусдорфовых пространствах
    $$\bigcap_{n \in \N} (\text{открытое, всюду плотное}) \quad \textbf{ -- всюду плотно}$$
\end{theorem}

\subsection{Теорема Банаха-Штейнгауза}
Сначала, введем определение
\begin{definition}
    Пусть $\G \subseteq \Hom(X,Y)$. Оно называется равностепенно непрерывным, если
    $$\text{для любой }\; W - \text{окр. 0 в Y существует }\; V - \text{окр. 0 в X: } \; \Gamma(V) \subseteq W$$
\end{definition}

Как мы увидим ниже, равностепенно-непрерывные семейства переводят ограниченные множества в ограниченные.
Теорема Банаха-Штейнгауза же скажет, что если точек, орбиты которых под действием $\Gamma$ ограниченны {\it много}, то $\Gamma$ равностепенно непрерывно.

\begin{theorem}
    Пусть $\Gamma \subseteq \Hom(X,Y)$ -- равностепенно непрерывно, а $E \subseteq X$ ограниченно.
    Тогда $\Gamma(E)$ суть ограниченное множество в $Y$.
\end{theorem}
\begin{proof}
Рассмотрим $W$ -- окрестность 0 в $Y$, выберем $V$ -- окрестность 0 в $X$ из определения. $E$ -- ограниченно $\Rightarrow$ имеем
$E \subseteq tV$ для больших $t$. Для них же:
    $$\Gamma(E) \subseteq \Gamma(tV) = t\Gamma(V) \subseteq t W$$
\end{proof}

\begin{theorem}[Банаха-Штейнгауза]
    Пусть $\Gamma \subseteq \CHOM(X,Y)$. Предположим, что множество $B = \{x \in X \; : \; \Gamma(x) - \text{ограниченно}\}$ имеет вторую категорию в $X$.
\begin{center}
    Тогда $B = X$ и $\Gamma$ равностепенно непрерывно.
\end{center}

\end{theorem}
\begin{proof}
Пусть $W$ -- уравновешенная окрестность 0 в $Y$, будем искать такую $V$ в $X$, что $\Gamma(V) \subseteq W$.
Для этого найдем такую $U$ -- уравновешенная окрестность 0 в $Y$, что $\overline{U} + \overline{U} \subseteq W,$ положим
$$E = \cap\limits_{\Lambda \in \Gamma}\Lambda^{-1}\big(\overline{U}\big).$$

$$x \in B \Rightarrow \Gamma(x) \in nU \text{ для больших }n \Rightarrow \frac{1}{n}x \in E \text{ для таких же }n$$
    Тогда заметим, что $B \subseteq \cup_n nE$. Значит, какой-то из $nE$ --
множество второй категории. Заметив, что $x \mapsto nx $ это гомеоморфизм $X$ получаем:
    $$\begin{rcases*}
E - \text{множество второй категории} \\
E - \text{замкнуто так как все $\Lambda \in \Gamma$ непрерывны}
\end{rcases*} \Rightarrow \text{в $E$ есть внутренняя точка } x_0$$
    Значит, в множестве $x_0 - E$ содержится окрестность нуля $V$, причем:
    $$\Lambda(V) \subset \Lambda x_0 - \Lambda(E) \subseteq \overline{U} - \overline{U} \subseteq W$$
    Тогда $\Gamma$ переводит ограниченные множества в ограниченные, а тогда $\Gamma(x)$ -- ограниченное для всякого $x \; \Rightarrow B = X$.
\end{proof}
\subsection{Полезные частные случаи}
\begin{theorem}
    Пусть $X$ -- $F$-пространство и $\Gamma \subseteq \CHOM(X,Y)$ и $\forall x \in X \; \Gamma(x)$ ограниченно в $Y$.
    Тогда $\Gamma$ равностепенно непрерывно.
\end{theorem}
\begin{proof}
По теореме Бэра, $F$-пространства являются множествами второй категории в себе.
\end{proof}
\begin{theorem}
    Пусть $X$ -- банахово, $Y$ -- нормируемо, причем $\sup_{\Lambda \in \Gamma} ||\Lambda x || < \infty$. Тогда существует такой $M > 0$, что
    $$||\Lambda x|| \leqslant M ||x||\quad \forall x \in X \; \forall \Lambda \in \Gamma$$
\end{theorem}
\begin{proof}
Применим предыдущую теорему к метрикам, порожденным нормами. В них ограниченность эквивалентна метрической.
\end{proof}
\begin{theorem}
    Пусть $\Lambda_n \in \CHOM(X,Y)$. Определим $$ C = \{x \in X\; : \; \Lambda_n x \text{ -- последовательность Коши}   \}; \quad  L = \{x \in X \; : \; \Lambda_n x \to \Lambda x \}.$$
    Тогда
    \begin{enumerate}
        \item Если $C$ второй категории в $X$, то $C = X$
        \item Если $L$ второй категории в $X$ и $Y$ -- $F$-пространство, то $L = X$ и $\Lambda$ -- непрерывно.
    \end{enumerate}
\end{theorem}
\begin{proof}\ \\
\begin{enumerate}
    \item Все последовательности Коши ограниченны $\Rightarrow $ по Б-Ш $\{\Lambda_n\}$ равностепенно непрерывна.
    Заметим, что $C \leqslant X$. Также, $C$ -- всюду плотное. \big({\small Пусть не так $\Rightarrow$ тогда $X = \overline{C} \oplus Y$, то есть $\overline{C}$ -- собственное подпространство.
    $\Rightarrow$ в нем нет внутренних точек $\Rightarrow$ оно I категории}\big).
    \ \\ Возьмем $x \in X$, будем показывать, что $\Lambda_n x$ -- последовательность Коши в $Y$, зафиксируем $W$ -- окрестность 0 в $Y$.
    По равностепенной непрерывности найдем симметричную $V$ -- окрестность 0 в $X$ -- такую, что ВСЕ $\Lambda_n(V) \subseteq W$.
    Так как $C$ всюду плотно, найдем $y \in C \cap \left(x+V \right)$. Тогда НСНМ
    $$(\Lambda_n - \Lambda_m)x = \Lambda_n(x-y) + (\Lambda_n - \Lambda_m)y + \Lambda_m(y-x) \in W + W + W$$
    \item $Y$ -- $F$-пространство $\Rightarrow$ $L = C$ из предыдущего пункта $\Rightarrow L = X$. Возьмем такие же, как и в прошлом пункте $W,V$. Для них $\Lambda_n(V) \subseteq W$ при ВСЕХ $n$. Тогда $\Lambda(V) \subseteq \overline{W}$, что эквивалентно непрерывности $\Lambda$.
\end{enumerate}
\end{proof}

\begin{theorem}
    $K \subseteq X$ -- выпуклый компакт, $\Gamma \subseteq \CHOM(X, Y)$, причем $\Gamma(x)$ -- ограниченно при всех $x \in K$. Тогда $\Gamma(K)$ -- ограниченное.
\end{theorem}

\subsection{Теорема об открытом отображении}
Пусть $f:S \to T$ -- отображение {\footnotesize ($S,T$ -- топологические, $f$ не обязательно непрерывна)} и $p \in S$.
Говорят, что $f$ открыта в $p$, если для любой $V$ -- окрестности $p$ образ $f(V)$ содержит окрестность $f(p)$.
\begin{theorem}
   Пусть $X$ -- $F$-пространство, $\Lambda \in \CHOM(X,Y)$, причем $\Lambda(X)$ второй категории. Тогда\\
   $$\text{(i)} \; \Lambda(X) = Y \;{\small \textcolor{red}{\Leftarrow}}\; \text{(ii)}\; \Lambda\text{ -- открыто} \quad \;\text{(iii)}\; Y \text{ -- $F$-пространство}$$
\end{theorem}
\begin{proof}
Стрелка << $\textcolor{red}{\Leftarrow}$>> объясняется тем, что $\Lambda(X)$ -- подпространство в $Y$.

Теперь пусть $V$ -- окрестность 0 в $X$, мы хотим проверить, что $\Lambda(V)$ содержит окрестность 0 в $Y$. Заведем метрику гарантированную тем, что $X$ это $F$-пространство $d$.

    Определим $$V_n = \left\{x\; | \; d(x,0) < \frac{r}{2^n}\right\}$$
    где $r$ такой маленький, что $V_0 \subset V$. Мы будем показывать, что $$W \underset{ (\textcolor{blue}{A})}{\subseteq} \overline{\Lambda(V_1)} \underset{ (   \textcolor{blue}{B}   )}{\subseteq} \Lambda(V)$$ для некоторой окрестности нуля $W$ в $Y$.

  (\textcolor{blue}{A}):  Знаем, что $$\overline{\Lambda(V_1)} \supseteq \overline{\Lambda(V_2) - \Lambda(V_2)} \supseteq \overline{\Lambda(V_2)} - \overline{\Lambda(V_2)}$$
    так как $V_2 - V_2 \subseteq V_1$. Поэтому мы будем показывать, что $\Int \overline{\Lambda(V_2)}$ непусто. Действительно
    $$\Lambda(X) = \cup_k k\Lambda(V_2) \; \Rightarrow \; \exists k : \; k\Lambda(V_2) \text{ -- II-категории} \Rightarrow \; \Lambda(V_2) \text{ -- II-категории} \; \Rightarrow \; \Int \overline{\Lambda(V_2)} \neq \varnothing$$


    (\textcolor{blue}{B}):  Зафиксируем $y_1 \in \overline{\Lambda(V_1)}$. Индуктивно построим последовательность $y_n \in \overline{\Lambda(V_n)} $ следующим образом.
    По тем же причинам, что и в пункте (\textcolor{blue}{A}), внутри $\overline{\Lambda(V_n)}$ содержится окрестность нуля $\forall n$. Следовательно,
    $$\left(y_n - \overline{\Lambda(V_{n+1})}\right) \cap \Lambda(V_n) \neq \varnothing,$$
    а значит мы можем найти такую $x_n \in V_n$, что $\Lambda x_n \in y_n - \overline{\Lambda(V_{n+1})}$. Вот и определим $y_{n+1} = y_n - \Lambda x_n$.
    $$d(x_n, 0) < 2^{-n} r \; \Rightarrow \; \sum_{k = 1}^n x_k\text{ -- посл-ть Коши} \; \Rightarrow \; \sum_{n = 1}^\infty x_n = x \in V_0 \subseteq V$$
    Также,
    $$\Lambda\left(\sum_{n = 1}^m x_n\right) = \sum_{n=1}^m \Lambda x_n = \sum_{n=1}^m (y_n-y_{n+1}) = y_1 - y_{m+1}\to y_1$$
Тут мы воспользовались непрерывностью $\Lambda$ говоря, что $y_{m+1} \to 0$ при $m \to \infty$. Получили то, что $y_1 = \Lambda x \in \Lambda(V_0) \subseteq \Lambda(V)$

    Теперь докажем (iii). Пусть $N = \ker \Lambda$. Мы знаем, что $X/N$ -- это $F$-пространство. Остаётся найти ТВП-изоморфизм $X/N \to Y$.
    Изоморфизм $f$ векторных пространств строится из первой теоремы о гомоморфизме (\smiley).
   \[ \begin{diagram}
\node{X} \arrow[2]{e,t}{\Lambda}
\arrow{se,b}{\pi}
\node[2]{Y} \\
\node[2]{X/N} \arrow{ne,r}{f}
\end{diagram}\]
    Проверим, что он -- гомеоморфизм.
\begin{enumerate}
    \item Если $V$ -- открыто в $Y$, то $f^{-1}(V) = \pi(\Lambda^{-1}(V))$ -- открыто, поскольку $\Lambda$ непрерывно, $\pi$ -- открыто.
    \item Если $E$ -- открыто в $X/N$, то $f(E) = \Lambda(\pi^{-1}(E))$ -- открыто, поскольку $\pi$ непрерывно и $\Lambda$ -- открыто.
\end{enumerate}

\end{proof}

\begin{center}
    \Large Следствия из нее
\end{center}

\begin{enumerate}
    \item \begin{enumerate}
        \item Если $X,Y$ -- $F$-пространства и $\Lambda \in \CHOM(X,Y) \qquad \Rightarrow \qquad \Lambda$ -- открыто
        \item Если к тому же $\Lambda$ -- биекция, то $\qquad \Rightarrow \qquad \Lambda^{-1}$ непрерывно.
    \end{enumerate}

    \item Если $X,Y$ -- Банаховы и $\Lambda \in \CHOM(X,Y)$ -- биекция, то $\exists c,C > 0$: для всех $x \in X$
    $$c||x|| \leqslant ||\Lambda x|| \leqslant C||x||$$
    \item  Если $\tau_1\subseteq \tau_2$ -- две векторые топологии на векторном пр-ве $X$ и оба $(X,\tau_1), (X,\tau_2)$ -- $F$-пространства, то $\tau_1 = \tau_2$.
\end{enumerate}




\subsection{Теорема о замкнутом графике}
Под графиком отображения $f: X \to Y$ имеется ввиду множество $\{(x,f(x))\}_{x \in X} \subseteq X \times Y$. Для непрерывных отображений в хаусдорфовы пространства график всегда замкнут, мы будем пытаться выяснить про какие-то факты, похожие на это.
Для начала обоснуем факт про замкнутность графика непрерывной функции:
\begin{remark}
Пусть $f:X \to Y$ непрерывна и $Y$ -- Хаусдорфово. Тогда график $f$ замкнут.
\end{remark}
\begin{proof}
Рассмотрим $(x_0, y_0)$ из дополнения графика в $X \times Y$, тогда отделим по хаусдофовости в $Y$ точки $y_0, f(x_0)$ окрестностями $U, V$ соответственно.
По непрерывности найдем $W$ -- окрестность $x_0$ в $X$ такую, что $f(W) \subseteq V$. А значит, $W \times U$ -- искомая окрестность $(x_0, y_0)$, содержащаяся в дополнении графика.
\end{proof}

\begin{theorem}[О замкнутом графике]
    Пусть $X,Y$ -- $F$-пространства, $\Lambda \in \Hom(X,Y)$ и его график замкнут. Тогда $\Lambda$ непрерывен.
\end{theorem}

\begin{proof}
    Рассмотрим $X\times Y$ как $F$-пространство.
График $\Lambda$ -- обозначим его $G$ -- замкнутое подпространство в $X\times Y$ (поскольку лямбда линейна). А значит, $G$ само по себе $F$-пространство
    Определим $$\pi_1: G \to X\; (x,\Lambda x) \mapsto x$$
    $$\pi_2: X\times Y \to Y\; (x,y) \mapsto y$$
    Тогда $\pi_1$ непрерывная линейная биекция $G\to X$, причем $G$ и $X$ -- $F$-пространства. Тогда по теореме об открытом отображении $\pi_1^{-1}$ непрерывно. А значит,
     $$\Lambda = \pi_2 \circ \pi_1^{-1} \; \text{непрерывна}$$
\end{proof}

\begin{remark}
Пусть для всяких $x_n \to x, \Lambda x_n \to y$ выполняется $y = \Lambda x$. Тогда график $\Lambda$ замкнут.
\end{remark}

\subsection{Билинейные отображения}
Пусть $X,Y,Z$ -- ТВП и $B: X\times Y \to Z$.

Можем определить $B_x:Y\to Z,\, B^y:X \to Z$ для фиксированных $x,y$ -- функции на срезах $X\times Y$.  Если они непрерывны, то $B$ называется \textbf{раздельно непрерывным}; если $B_x, \, B^y$ линейны, то $B$ называется \textbf{билинейным}.
В некоторых случаях из раздельной непрерывности следует обычная непрерывность:
\begin{theorem}
    Если $X$ -- $F$-пространство и $B$ -- раздельно непрерывное билинейное. Тогда $B$ секвенциально непрерывно. В частности, если $Y$ метризуемо, то $B$ непрерывно.
\end{theorem}
\begin{proof}
    Выберем $x_n \to x_0$ в $X$ и $y_n \to y_0$ в $Y$
Возьмем $U, W$ окрестности 0 в $Z$ такие, что $U+U \subseteq W$, положим $b_n(x) = B(x,y_n)$.
    \begin{enumerate}
        \item  Так как эти последовательности сходятся, множества $\{b_n(x)\}$ ограничены в $Z$.
        \item  Тогда $b_n(x)$ --  непрерывные линейные отображения из $F$-пространства $X$ в $Z$.
        \item  Значит, по \ref{}[следствию из теоремы Банаха-Штейнгауза], семейство $\{b_n\}$ равномерно непрерывно.
    \end{enumerate}
    А значит найдется $V$ -- окрестность 0 в $X$ такая, что $\forall n\; b_n(V) \subset U$. Тогда начиная с некоторого места:
    $$B(x_n, y_n) - B(x_0,y_0) = b_n(x_n-x_0)+B(x_0,y_n-y_0) \in U + U \subseteq W$$
    Если $Y$ метризуемо, то $X\times Y$ метризуемо, а значит секвенциальная непрерывность эквивалентна обычной
\end{proof}



\section{Выпуклость}
Обозначим $X^*:= \CHOM(X, \RR)$, на этом множестве очевидно есть структура векторного пространства. Для следующей теоремы не нужна никакая топология.
\begin{theorem}\label{CB}[Хана-Банаха]
    Пусть $X$ -- $\R$-векторное пространство, $M\leqslant X$. И даны:
    \begin{enumerate}
        \item $p : X\to \R$ -- функция, удовлетворяющая условиям
        $$p(x+y) \leqslant p(x) + p(y),\qquad p(tx) = tp(x) \qquad \forall x,y\in X, \forall t > 0$$
        \item $f \in \Hom(M, \R)$ -- функционал такой, что $f \leqslant p$ на $M$.
    \end{enumerate}
    Тогда существует $\Lambda \in \Hom(X, \R)$ такой, что
    $$\Lambda|_M = f, \qquad -p(-x) \leqslant \Lambda x \leqslant p(x), \;\forall x \in X$$
\end{theorem}
\begin{proof}
Пусть $M \neq X$, возьмем $x_1 \in X \backslash M$, положим $M_1 = M + \Span(x_1)$. Научимся продолжать $f$ до $f_1$ на $M_1$.
    $$f(x)+f(y) = f(x+y) \leqslant p(x+y) \leqslant p(x-x_1)+p(x_1-y)$$
    $\Rightarrow$
    $$\textcolor{blue}{ f(x) - p(x-x_1)} \leqslant p(y+x_1) - f(y)\quad  \forall x,y \in M$$
    Тогда положим $$\alpha:= \sup_{x \in M}\textcolor{blue}{ f(x) - p(x-x_1)},$$
    Определим $f(x+tx_1) = f(x) + t\alpha$ -- функционал на $M_1$, причем $f_1 \leqslant p$ в силу неравенств  $f(x) - \alpha \leqslant p(x-x_1)$ и $f(y) + \alpha \leqslant p(y+x_1)$..

    Теперь будем аксиомой выбора продолжать всё это дело до $X$. А именно, мы воспользуемся теоремой Хаусдорфа. Определим $$\cP = \big\{(M', f'): M \leqslant M' \leqslant X, \; f'\text{ -- функционал на $M'$: } f'|_M = f, \; f' \leqslant p\big\}$$
    На $\cP$ задан частичный порядок:   $$(M', f')\preccurlyeq (M'', f'') \; \Leftrightarrow \; (M' \leqslant M'')\;\wedge \; f''|_{M'} = f' $$
    По теореме Хаусдорфа в $\cP$ существует максимальное линейно-упорядоченное множество $\Omega$. Положим
    $$\Phi = \big\{ M'\; : \; \exists f'\text{ -- функционал на $M'$ такой, что} (M', f') \in \Omega\big\}$$
    Тогда $\Phi$ линейно упорядочено относительно $\subseteq$, а значит
    $$\tilde M = \cup\{M'\; : \; M' \in \Phi\}\textbf{ -- подпространство в } X$$
    Понятным образом продолжим $f$ до  $\Lambda$ -- функционала на $\tilde M$, подчиненного $p$. Также видно, что $\tilde M$ не может быть собственным подпространством $X$. Из линейности следует нужное неравенство.
\end{proof}
\begin{theorem}[Хана-Банаха x2]
    Пусть $M \leqslant X$, $p$ -- полунорма на $X$, $f$ -- функционал на $M$ такой, что
    $$|f| \leqslant p \quad \text{на }M$$
    Тогда $f$ можно продолжить до $\Lambda$ -- функционала на $X$ такого, что
    $$|\Lambda|\leqslant p \quad \text{на }X$$
\end{theorem}
\begin{proof}
Если $X$ -- это $\R$-пространство, то эта теорема следует из предыдущей. Если же это $\C$-пространство, положим $u = \Re f$.
\newpage
\begin{center}

    По (\ref{CB}) $u$ продолжается до $U \in \Hom_\R(X, \R)$  подчиненного $p$ всюду.
    $$\Downarrow$$
    Если $\Lambda \in \Hom_\C(X, \C)$ -- такой, что $\Re \Lambda = U$, то $\Lambda|_M = f$
\end{center}

\end{proof}


    \bibliography{main}
    \bibliographystyle{plain}

\end{document}
